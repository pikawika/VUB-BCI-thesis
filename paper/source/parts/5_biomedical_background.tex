% TODO:
%   - Make this background chapter
%   - Section on ethics
%   - Split out your own work from literature
%   - Move BCI pipeline to separate chapter
% ----------  
% Questions:
%   - XXX

% https://www.brainlatam.com/blog/wet-dry-active-and-passive-electrodes.-what-are-they-and-what-to-choose-413
% https://www.brainlatam.com/blog/a-brief-introduction-to-eeg-and-the-types-of-electrodes-75

% Goed boek voor sources bci_review_book_chapter

% In a new chapter, reset the GLS to once again use full version in first occurence
\glsresetall

\chapter{Biomedical signals}
\label{ch:biomedical_signals}

% ---------------------------------------------- 

\section{Better understanding the data used by BCIs}
\label{sec:biomedical_signals_why}

% TODO
\lipsum[1-2]

% ---------------------------------------------- 

\section{Origin of brain-signals}
\label{sec:biomedical_signals_origin}

% TODO
\lipsum[1-2]

% - - - - - - - - - -

\subsection{Bioelectricity in the human body}
\label{subsec:biomedical_signals_origin_bioelectricity}

% TODO
\lipsum[1-4]

% - - - - - - - - - -

\subsection{Measurable brain activity}
\label{subsec:biomedical_signals_origin_brain_activity}

% TODO
\lipsum[1-3]


% - - - - - - - - - -

\subsection{Anatomy of the brain}
\label{subsec:biomedical_signals_origin_anatomy_brain}

% TODO
\lipsum[1-4]

% - - - - - - - - - -

\subsection{Neuroplasticity and inter-human variation}
\label{subsec:biomedical_signals_origin_general_brain_layout}

% TODO
\lipsum[1-3]

% ---------------------------------------------- 

\section{Types of brain-signals}
\label{sec:biomedical_signals_type_of_signals}

% TODO 'bci paradigms" 
% info about signals: https://www.e-iji.net/dosyalar/iji_2021_2_48.pdf
\lipsum[1-2]

% - - - - - - - - - -

\subsection{Brain waves frequencies}
\label{subsec:biomedical_signals_type_of_signals_brain_wave}

% TODO
\lipsum[1-2]

% - - - - - - - - - -

\subsection{Event related potentials}
\label{subsec:biomedical_signals_type_of_signals_erp}

% TODO
% P300 en andere signalen vermeld in introduction
\lipsum[1-2]

% - - - - - - - - - -

\subsection{Motor imagery}
\label{subsec:biomedical_signals_type_of_signals_motor_imagery}

% TODO zie chapter 23 bci_handbook
% Issues MI: https://www.frontiersin.org/articles/10.3389/fnins.2021.824759/full
\Gls{mi} is the process in which a person generates brain-activity in the motor cortex merely by imagining motor movements.
\Gls{mi}-based \glspl{bci} are interesting because they don't require any external stimulus nor effective motor movements

\lipsum[1-3]

% ---------------------------------------------- 


\section{Measuring brain-signals}
\label{sec:biomedical_signals_measuring}

% TODO: Different affordable consumer-grade EEG devices have appeared in both Academia (e.g. [5, 6, 7, 8]) and the market (e.g. B-Alert X10, NeuroSky, OpenBCI, Emotiv)
% FROM: cheap_bci_feasibility

Many comparisons between different types of measuring equipment, often with greatly differing costs, have already been made \citep{bci_cheap_viability1, bci_cheap_viability2, bci_cheap_viability3, bci_cheap_viability4, bci_cheap_viability5}.
The main consensus is that the cheaper consumer-grade equipment has the potential to reach similar performance of a conventional, often medical-grade, BCI system.
These results are promising but due to the controlled nature of the experiments, they might not reflect real-life applications accurately.
As discussed before, the user experience of a \gls{bci} system is as important if not more important then the raw performance of the system.
% FROM: cheap_bci_feasibility
% OpenBCI1 is an open-source, versatile and affordable biosensing system which can be used to acquire not only EEG signals but also to measure electrical activity of muscle (EMG) and heart (ECG). All OpenBCI boards are based on the open-source electronic platform Arduino with wireless connection to the computer. OpenBCI offers a variety of low-cost amplifiers (boards), electrode systems (e.g. 3D-printed headware) and a software for viewing and recording the biosignals (OpenBCI GUI).2
% as well as from the user’s comfort perspective [4, 13, 14].

% ---------------------------------------------- 

\subsection{Measuring modalities}
\label{subsec:biomedical_signals_measuring_modalities}
% EEG ECOG ETC ETC

% TODO
\lipsum[1-5]

Research by \citet{human_eeg_discovery} is the first in describing the measurement of brain waves from the human skull in a non-invasive manner.
Because of this, the German neuroscientist and psychiatrist Hans Berger is often seen as the inventor of \gls{eeg}.
Whilst he was one of the first to use the term \textit{elektrenkephalogramm}, it was Richard Caton who first described the findings of brain waves in general.
He found this phenomena in animal brains as early as 1875 \citep{first_eeg}.
Since then, \gls{eeg} methodology and equipment has matured and evolved a lot.

% - - - - - - - - - -

\subsection{Standardized EEG measuring systems}
\label{subsec:biomedical_signals_measuring_standardization}

% TODO
\lipsum[1-4]

% - - - - - - - - - -

\subsection{Comparison of available EEG measuring equipment}
\label{subsec:biomedical_signals_measuring_equipment}

% TODO
\lipsum[1-5]

% https://imotions.com/blog/eeg-headset-prices/
% Nextmind 
% Neurosky
% Interaxon
% Muse
% Emotiv
% myBrain
% OpenBCI

% TODO: citing for sources of img?
\begin{figure}[ht]
  \begin{minipage}{\textwidth}
    \centering
    \begin{subfigure}{.48\textwidth}
        \centering
        \includegraphics[width=\textwidth]{images/hardware/berger_hardware.png}
        \captionsetup{width=0.9\linewidth}
        \captionsetup{justification=centering}
        \caption{Experimental analog \gls{eeg} recording equipment used by \citet{human_eeg_discovery}.\\Figure by by \citet{oldest_eeg_hardware}.}
        \label{fig:eeg_hardware_evolution_1}
    \end{subfigure}
    \hfill
    \begin{subfigure}{.48\textwidth}
        \centering
        \includegraphics[width=\textwidth]{images/hardware/eeg_1950.jpg}
        \captionsetup{width=0.9\linewidth}
        \captionsetup{justification=centering}
        \caption{Medical-grade analog \gls{eeg} recording equipment estimated to be from the 1950's. \\Figure by Devotor\footnotemark[1].}
        \label{fig:eeg_hardware_evolution_2}
    \end{subfigure}
    \captionsetup{width=0.9\linewidth}
    \bigskip
    \begin{subfigure}{.48\textwidth}
        \centering
        \includegraphics[width=\textwidth]{images/hardware/early_portable_eeg.jpg}
        \captionsetup{width=0.9\linewidth}
        \captionsetup{justification=centering}
        \caption{Early portable analog EEG recording equipment from the late 1950's. \\Figure by Sam Brusco\footnotemark[2]. }
        \label{fig:eeg_hardware_evolution_3}
    \end{subfigure}
    \hfill
    \begin{subfigure}{.48\textwidth}
        \centering
        \includegraphics[width=\textwidth]{images/hardware/grey_walter.jpg}
        \captionsetup{width=0.9\linewidth}
        \captionsetup{justification=centering}
        \caption{William Grey Walter and medical-grade analog \gls{eeg} recording equipment, 1964.\\Figure by Burden Neurological Institute\footnotemark[3].}
        \label{fig:eeg_hardware_evolution_4}
    \end{subfigure}
    \captionsetup{width=0.9\linewidth}
    \captionsetup{justification=centering}
    \caption{Early analog \gls{eeg} equipment.}
    \footnotetext[1]{\url{https://www.charismaticplanet.com/the-electroencephalogram-1924/}}
    \footnotetext[2]{\url{https://www.medicaldesignandoutsourcing.com/medtech-memoirs-the-electroencephalograph-eeg/}}
    \footnotetext[3]{\url{http://dx.doi.org/10.15180/181003/019}}
    \label{fig:eeg_hardware_early_analog}
  \end{minipage}  
\end{figure}

% - - - - - - - - - -

\subsection{Usages of invasive systems}
\label{subsec:biomedical_signals_measuring_invasive}

% TODO
\lipsum[1-3]

% - - - - - - - - - -

\subsection{Common EEG artefacts}
\label{subsec:biomedical_signals_measuring_artefacts}

% TODO also discuss how to use them e.g. eye blink as input
\lipsum[1-4]


% ---------------------------------------------- 

\section{Using non-invasive EEG measurements for MI classification}
\label{sec:biomedical_signals_eeg_for_mi_classification}

% TODO
\lipsum[1-2]