% TODO:
%   - See cheap_bci_feasibility for an example of good abstract 
% ----------  
% Questions:
%   - Should this be included in ToC?

\chapter{Abstract}
\label{ch:abstract}

This master thesis explores the field of \glspl{bci}.
First, it aims to provide a great foundation for the knowledge required for working in the \gls{bci} field as a computer scientist.
To accomplish this, an exhaustive literature review in the introductory chapter aims to provide a great general introduction to the field and current state-of-the-art as well as challenges and promises of the field.
A chapter on \glspl{biosignal}, the source of data for \gls{bci} systems, is also provided.
It discusses how \gls{eeg} can be measured and provides an overview of common hardware, issues and more.

Next, the viability of real-world applications using classification algorithms on live \gls{eeg} measures collected from affordable \glspl{bci} hardware is explored.
This is done by first introducing a general \gls{bci} pipeline and discussing all of its components.
Afterwards, a three-signal control system is proposed as \gls{poc} based on this general \gls{bci} pipeline.
Special care is given to include all important details of the system, in an attempt to improve reproducibility.
The system is also evaluated taking into account best-practice techniques whilst also realising the \gls{bci} field lacks standardized testing strategies.

% TODO
\textbf{TODO: this abstract should be further completed after the thesis is finished.}
