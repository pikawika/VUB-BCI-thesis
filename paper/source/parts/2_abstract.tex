% TODO:
%   - Nalezen op einde of alles included; mss ook extensie niet gewoon comparison
% ----------  
% Questions:
%   - 

% NOTE: In a new chapter, reset the GLS to once again use the full version in the first occurrence
\glsresetall

\chapter*{Abstract}
\label{ch:abstract}

\Glspl{bci} are fascinating systems that aim to translate brain signals into actions.
A recent rise in scientific and commercial interest in these systems has helped in popularizing them outside highly specialized labs.
With seemingly thousands of use cases for these systems, many researchers and students have shown interest in contributing to the field of \glspl{bci}.
However, the highly interdisciplinary nature of \gls{bci} research among other factors cause a steep learning curve which deters many individuals who wish to work with these kinds of systems.
Those who do commit to overcoming the steep learning curve are faced with complex and sometimes contradictory research.

This master thesis aims to facilitate this entry into the fascinating field of \glspl{bci} for computer scientists.
Part \ref{part:understanding} aims to provide all foundational knowledge required to work in the \gls{bci} field as a computer scientist.
An exhaustive literature review in the introductory chapter aims to provide an intuitive introduction to the field by addressing the reasons for a growing scientific and commercial interest along with some of the common use cases for \glspl{bci}.
This introductory chapter also addresses some promising opportunities and obstacles in the field.
The introductory chapter ends by touching upon some ethical challenges related to \gls{bci} research.
The other chapters in Part \ref{part:understanding} provide the more technical knowledge required for computer scientists in the field.
Chapter \ref{ch:biomedical_signals} discusses the origin and measuring modalities for \glspl{biosignal} and brain signals in particular.
Chapter \ref{ch:processing_signals} goes over some common \gls{bci} pipeline components for classifying  \gls{mi} \gls{eeg} data, the most common task of computer scientists in this field.

Part \ref{part:development} puts this theory to practice and compares multiple state-of-the-art techniques in \gls{mi} \gls{eeg} classification.
The focus of this part is on offline techniques with both two-step \gls{ml} approaches and one-step \gls{dl} approaches implemented.
Some of the steps required to move to an online classification system and eventually a complete \gls{bci} system are also addressed.
This master thesis ends with Part \ref{part:reflection} where the results and added value of this work are reflected upon.



