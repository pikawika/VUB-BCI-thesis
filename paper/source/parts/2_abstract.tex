% TODO:
%   - Nalezen op einde of alles included; mss ook extensie niet gewoon comparison
% ----------  
% Questions:
%   - Should this be included in ToC?

\chapter{Abstract}
\label{ch:abstract}

\textbf{NOTE: this abstract could be altered once the thesis is finished.}

This master's thesis explores the field of \glspl{bci}.
Part \ref{part:understanding} aims to provide a great foundation of the knowledge required to work in the \gls{bci} field as a computer scientist.
An exhaustive literature review in the introductory chapter aims to provide a more intuitive introduction to the field by addressing the reasons for a growing scientist and commercial interest along with some of the common use cases for \glspl{bci}.
This introductory chapter also addresses some promising opportunities and obstacles in the field of \glspl{bci}.
The introductory chapter ends by touching upon some ethical challenges related to \gls{bci} research.
Chapters \ref{ch:biomedical_signals} and \ref{ch:processing_signals} from part \ref{part:understanding} are more technical and tailored towards computer scientists.
Chapter \ref{ch:biomedical_signals} discusses the origin and measuring modalities for \glspl{biosignal} and brain signals in particular.
Chapter \ref{ch:processing_signals} goes over some common \gls{bci} pipeline components for classifying  \gls{mi} \gls{eeg} data.

Part \ref{part:development} puts this theory to practice and compares multiple state-of-the-art techniques in \gls{mi} \gls{eeg} classification.
The focus of this part is on offline techniques and both two-step \gls{ml} approaches and one-step \gls{dl} approaches.
However, some of the steps required to move to an online classification system and eventually a complete \gls{bci} system are also discussed.
This master's thesis ends with Part \ref{part:reflection} where the added value and results of this work are reflected upon.



