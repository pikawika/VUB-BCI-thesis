\glsresetall

\chapter*{Abstract}
\label{ch:abstract}

\Glspl{bci} are systems that aim to translate brain signals into actions.
A recent rise in both scientific and commercial interest in these systems has helped popularize them outside highly specialized labs.
However, the highly interdisciplinary nature of \gls{bci} research among many other complex factors results in a steep learning curve which deters many individuals who wish to work with these kinds of systems.

This master thesis aims to facilitate this entry into the \gls{bci} field for computer scientists.
Through an extensive but intuitive introduction of \glspl{bci}, a scoping review is performed to address recent advancements, current opportunities and open issues in the field together with some ethical concerns.
This intuitive understanding is then grounded by providing the required theoretical and technical background on the \glspl{biosignal}, brain signal acquisition systems and the general \gls{mi} \gls{eeg} classification pipeline that enables these systems.

This theory is put to practice by discussing, implementing and evaluating seven unique offline \gls{mi} \gls{eeg} classification pipelines.
In particular, two traditional two-step \gls{ml} approaches based on \gls{csp} feature extraction, three literature proposed state-of-the-art \gls{cnn} based approaches and two proposed extensions providing \gls{lstm} functionalities to such \gls{cnn} models are considered.
The latter extensions were proposed to explore the question if added \gls{lstm} functionalities helps the state-of-the-art EEGNet model in optimally decoding \gls{mi} \gls{eeg} data.
It was found that for 1.5-second windowed samples centred around a known event, one of the \gls{lstm} extension outperforms the \gls{cnn} based state-of-the-art EEGNet model it is based on in an intersubject evaluation strategy for testing three-class \gls{mi} \gls{eeg} classification performance.
This model obtained a mean test accuracy of  $65.52$ ± $0.89$, over the $60.68$ ± $1.64$ obtained by the EEGNet model it is based on.
Other \gls{cm} derived metrics were also found favourable for the extension.
It is discussed how this difference could increase further when using a different windowing technique.

Due to the considerable amount of information this master thesis wishes to provide, a multitude of new questions and potential future work arises.
Some of these discussed future possibilities, such as moving from the offline \gls{mi} \gls{eeg} classification pipelines to a complete \gls{bci} system, benefit from pilot studies also presented in this master thesis.
In this regard, a pilot study found that an intersubject test accuracy of over 90\% can be reached for certain subjects of the open-source dataset used.