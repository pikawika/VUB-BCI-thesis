\chapter{Conclusions, future work and self-reflection}
\label{ch:discussion}

% ---------------------------------------------- 
% INTRODUCTION
% ---------------------------------------------- 
\section{Introduction}
\label{sec:discussion_introduction}
% NOTE: "Introduction" exists in each chapter and gives a short intro to the chapter + what can be expected in the chapter

This master thesis set out two major goals, to provide a document that facilitates the entry into the \gls{bci} field for computer scientists and to evaluate different types of offline \gls{mi} \gls{eeg} classification pipelines, especially \gls{cnn} based approaches and the proposed \gls{lstm} extensions.
The latter aimed to answer the research question if adding \gls{lstm} functionality to a state-of-the-art \gls{cnn} \gls{mi} \gls{eeg} classification pipeline can further improve its performance.
This chapter reviews to what degree these goals were accomplished and what the found answer to this research question is.
Some future work that emerged from the work done in this master thesis is also addressed and the master thesis ends with a personal endnote. 


% ---------------------------------------------- 
% usability
% ---------------------------------------------- 
\section{Are brain-computer interfaces made easy?}
\label{sec:discussion_are_bci_easy}

One of the two major goals of this master thesis was to facilitate the entry into the \gls{bci} field for computer scientists as a steep learning curve deters many that are interested.
In this regard, the first chapter of this master thesis is a rather unique approach to accomplishing this goal.
Realising that many excellent systematic review articles on the \gls{bci} field and related topics already exist, it was chosen to provide a more intuitive introductory chapter into \gls{bci} by focusing on some of the field's history, present and future whilst referencing to resources providing a more formal and systematic overview.

In this regard, research from the late 1990s and onwards was studied to derive the most important components in the cycle of increasing popularity for \gls{bci} research.
A clear upwards trend in publications started around 2000, with researchers primarily focused on using traditional two-step \gls{ml} in combination with \gls{csp} or one of its many extensions to classify many different types of brain signals.
As \gls{dl} saw a spike in popularity, so did its application to brain signals.
This is one of many discussed reasons that the 2010s saw an explosion in both research and commercial interest of \glspl{bci}.
Taking into account the many use cases for \glspl{bci} discussed in this master thesis, the appeal of a general \gls{bci} system is easy to see.
However, some obstacles and open issues are still in the way for \gls{bci} usage to see true widespread use in the real world.
These challenges and some of the opportunities were therefore also discussed, together with a note on some ethical questions.

From these rather intuitive explanations, a shift was made to a more theoretical and technical description of brain signals, measuring modalities and some of the standards and guidelines used when working with brain signals and \gls{eeg} in particular.
The focus then shifted from a neurological point of view on brain signals and data acquisition to a computer scientist point of view on the classification pipelines \gls{bci} systems use.
A general pipeline for classifying brain signals with traditional two-step \gls{ml} and one-step \gls{dl} was discussed in detail, with provided examples for each component.
This included high-level descriptions visualized in easy-to-understand figures but also mathematical concepts needed to understand the working of multiple preprocessing steps and \gls{ml} and \gls{dl} concepts.
Some practical points for a computer scientist working on such a brain signal classification pipeline were also given.
This included how to perform correct model evaluation and some of the many potential pitfalls when doing such research.

The remainder of this master thesis was then focused primarily on the evaluation of multiple proposed classification pipelines, which also helps in further understanding how the previously mentioned theoretical and conceptual components translate to an effective pipeline.
Taking all of these things into account, the author of this master thesis believes the goal of providing a low-entry document to the \gls{bci} field from a computer scientist's point of view is well achieved.
Being made by a computer science student that had to go through the many ups and downs in the steep learning curve of the field, it was easier to see the gaps in more intuitive and fundamental background knowledge that was missing from scientifically far superior systematic review articles.
therefore calling this first part of the thesis equal to those articles would be unfair to both the articles and this master thesis.
It is a rather unique document where initially a more intuitive approach to explaining some of the difficult components was taken but later on the document switched back to its scientific roots to form what can be considered a scoping review for the \gls{bci} field.

% ---------------------------------------------- 
% Added value
% ---------------------------------------------- 
\section{Is providing LSTM functionalities to a CNN model beneficial?}
\label{sec:discussion_lstm_cnn_benefit}

The second major goal of this master thesis was to evaluate different types of offline \gls{mi} \gls{eeg} classification pipelines, especially \gls{cnn} based approaches and to propose a useful extension to one of these state-of-the-art models.
In practice, this meant two traditional two-step \gls{ml} approaches using \gls{csp} and three state-of-the-art one-step \gls{dl} models based on a \gls{cnn} architecture.
In particular, the literature proposed \gls{dl} architectures were EEGNet \citep[as proposed by ][]{eeg_model_eegnet}, ShallowConvNet and DeepConvNet \citep[both proposed by ][]{eeg_model_hbm}.
After a detailed discussion of these architectures, it was discussed how the time series nature of \gls{eeg} could potentially be an ideal application for \gls{lstm}.

Rather than proposing an \gls{ann} based on \gls{lstm} alone, it was explored if adding this functionality to an existing proven to be successful \gls{cnn} model can further boost its performance.
One extension used a regular \gls{lstm} layer after an otherwise almost equal to EEGNet architecture, whilst another replaced the second half of EEGNet with a convolutional \gls{lstm} arrangement.
initially, the latter was thought to be a more interesting design as it directly integrated into the EEGNet proposed architecture.
However, the Keras-provided implementation of this convolutional \gls{lstm} layer does not support CUDA optimisation and as such proved to be computationally heavy to train. 
This made the already computationally hard process of finding a good network architecture and parameters even harder with waiting times in the hours to get any insight on how the changes affect validation performance.
Adding to this, a prediction time test showed that the prediction using this convolutional \gls{lstm} based model was more than 20 times slower than EEGNet and the regular \gls{lstm} extended EEGNet.

This rather practical issue does make it that it is unlikely this network architecture will be studied further, as it got comparable to or negligible better results than the EEGNet model it was based on.
Luckily, intersubject testing using a 1.5-second window centred around a known event showed promising results for the regular \gls{lstm} extension to EEGNet.
A mean test accuracy of  $65.52$ ± $0.89$, over the $60.68$ ± $1.64$ obtained by the EEGNet model was found.
Whilst the standard deviations don't overlap, it is hard to call this definitive proof that adding \gls{lstm} functionality boosts the performance of an EEGNet. 
As such, the research question remains an open one, with data that is suggestive of a positive answer.

% ---------------------------------------------- 
% Future work
% ---------------------------------------------- 
\section{Future work}
\label{sec:discussion_future_work}

Multiple future extensions on this master thesis are possible, some of which have already been discussed throughout this master thesis.
Perhaps the most suitable future work would be to use the proposed regular bidirectional \gls{lstm} extended EEGNet model in an equal test environment but using overlapping sliding windows.
Not being centred around the event, and thus not having the \gls{ers} and/or \gls{erd} following the \gls{mi} task at a somewhat fixed time, it is expected that EEGNet would perform worse compared to the results noted here.
In this regard, the added \gls{lstm} layer is expected to work better with processing the time series as it has been built around the idea of being used in sequential data where some time points might be more important than others.
This makes it rather likely that a more significant difference will be noticed and that a formal answer to the research question of this master thesis can be formed.

The sliding window approach would also make it more suitable for use in an online \gls{bci} system.
As such, the logical next step after a positive finding would be to train the model on all available data of the CLA dataset and perhaps even some of the other \gls{mi} datasets provided by \citet{eeg_data} that contain left-hand, right-hand and passive \gls{mi} tasks.
A pilot study already showed that this can increase intersubject performance significantly.
The model could then be compared to existing \gls{lstm} and \gls{cnn}-\gls{lstm} models proposed in the literature as well as the EEGNet and DeepConvNet model from this paper.
The ShallowConvNet model may be considered as well, although the author of this paper believes EEGNet to be a more attractive variant of a \gls{dl} model that is inspired on \gls{fbcsp}.
A different dataset containing more \gls{mi} classes may also be considered, although this will likely degrade the accuracy and risky errors, potentially making it unsuitable for online \gls{bci} use.
Testing different thresholds for classifying the samples as being of passive \gls{mi} is also an option that is worthwhile to look into.
As discussed, having a low accuracy but better risky misclassification rate is desired for \gls{bci} usage and is possible to obtain by changing this threshold.

For those interested in the visualisation of these models, and given the discussed law proposals surrounding explainable \gls{ai} perhaps everyone should be interested, looking into the discussed potentials of visualising the model is also a worthwhile route.
This should then be combined by subsampling the channels as proposed in this master thesis to only include those channels that have valuable information surrounding the \gls{mi} tasks.
Besides being more compliant with explainable \gls{ai} principles, the discussed medical applications of \gls{bci} surrounding neuroplasticity and rehabilitation can directly use such visualisation in their day-to-day life.
There are many other potential future works, but if there is one thing this master thesis aimed to do it is to illustrate how fascinating and full of possibilities the \gls{bci} field is and as such, the interested reader most likely has obtained multiple ideas themself from simply reading this master thesis.

% ---------------------------------------------- 
% Conclusions
% ---------------------------------------------- 
\section{Final remarks and a personal endnote}
\label{sec:discussion_final_remarks}

Having looked back at this master thesis that passed the three-digit page numeration, this final section addresses some self-reflection on this work.
Three major things became apparent whilst working on this master thesis.
First, to say it with a nice quote, \textit{the more you know, the more you know you don't know}.
The steep learning curve of the \gls{bci} field combined with a high interest in the field resulted in four papers being added to the to-read list whilst only one was removed.
This resulted in an over-the-top literature review, with more than 300 consulted sources.
Whilst the page count of this master thesis was originally even higher, it should have probably been more compact but an unlikely combination of events caused a significant shortage in time to downsize this work qualitatively and it was chosen to let Chapter 1 as is.
Second, whilst a stressful process at times, it made me discover a passion for performing research I didn't know I had.
If it were not for time constraints, I would've happily worked on this master thesis for another few months.
Third, I want to end this master thesis similarly to how it started.
I want to end it by thanking my supervisor Professor Doctor Geraint Wiggins and my advisor Arnau Dillen for not only allowing me to perform this research but for being understanding of some of the unfortunate events that occurred along the way.