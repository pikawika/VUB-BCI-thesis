% TODO:
%   - References in section helping people with disabilities for examples
%   - More complete and referenced 1.1
%   - Longer more descriptive titles
% ----------  
% Questions:
%   - Should we also use parts or not?

% Uncomment this if use of parts is desired
%\part{First part}

\chapter{Introduction}
\label{ch:introduction}

\Glspl{bci} and \glspl{bmi} are starting to gather more interest from the general public.
These systems, consisting of hardware and software, aim to read and stimulate a user's biomedical signals for a wide range of purposes.
Neuralink, an Elon Musk company, helped to popularize them outside of the research field.
Neuralink's initial white paper discusses its aim to create a scalable high-bandwidth \acrshort{bmi} system, focusing on its mechanical achievements \citep{neuralink_whitepaper}.
Only a mere year later, the company held a conference with a live demo of their \acrshort{bmi} implanted into the skulls of pigs.

Whilst such novel interaction application have yet to see wide spread market adoption, \acrshortpl{bci} have been studied for decennia by researchers such as \citet{early_bci}.
An in-depth history from \acrshortpl{bci} can be found from Andrea \citet{bci_history}.
\citet{bci_review} gives an overview of the different kinds of \acrshortpl{bci} and corresponding techniques to be used.
These sources give a great high-level insight into the technology and terminology used in this field.
A good introductory book on \acrshortpl{bci} from this same period is one from the well-known Professor in this field: Jonathan Wolpaw \citep{bci_book}.

Whilst techniques used today remain relatively similar to those discussed in these older sources, the hardware complexity has been evolving drastically.
An important issue researchers keep getting confronted with is the fact that there's still a lot of mysteries about the brain's inner working, leaving many of the finer intercepted brain signals unexplained.
\citet{brainmapping} states that the progress of mapping the brain and it's billions of sensors and connections is accelerating yet still far from finished.
This mapping would be a huge step in understanding the brain.

Luckily, novel \gls{ml} techniques can help with processing and reasoning on the data these systems collect.
Especially due to recent developments in \gls{dl} and \gls{nn}, \acrshortpl{bci} are being used more in treating neurological diseases \citep{bci_diseases} but also more commercial applications, like a brain-controlled "virtual keyboard" as discussed by \citet{bci_keyboard}.

A challenge with \acrshort{dl} applications in general, and certainly with \acrshortpl{bci}, is the fact that training a deep learner in an unsupervised manner can take a lot of data and time and be unpredictable due to it's black box principle.
These things are unwished-for in \acrshort{bci} systems, especially when not only reading but also stimulating brain signals.
In general, full supervised learning isn't possible either due to the lack of understanding the brain signals.
Therefore, low-confidence labeled data is often used in a semi-supervised fashion as explained by \citet{deep_learn_low_label}.

\section{Gaining popularity}
\label{sec:introduction_gaining_popularity}

As was already discussed, Neuralink, a company by Elon Musk aiming to provide an invasive \acrshort{bci} to be used by the masses as a mean for fast communication between human and machine, has put the research field in new daylight.
However, the reason \acrshortpl{bci} applications are gaining more and more interest follows from a wide variety of reasons.
The following section will discuss the most important ones.

\subsection{Commercialisation by big tech}
\label{subsec:introduction_gaining_popularity_big_tech}
TODO

\subsection{Improved machine learning}
\label{subsec:introduction_gaining_popularity_improved_ml}
TODO

\subsection{More affordable hardware}
\label{subsec:introduction_gaining_popularity_affordability}
TODO

\subsection{Boost in efficiency}
\label{subsec:introduction_gaining_popularity_efficiency}
TODO

\section{Helping people with disabilities}
\label{sec:introduction_helping_disabled}

Whilst the commercial interest of \acrshortpl{bci} is apparent, it is not the only reason these systems are gaining popularity.
As technology evolves, it has often been the case that people with certain disabilities benefit from the evolution as well.
Recently, image and speech to text technology has found its way directly into smartphone operating systems.
Whilst it is handy for most to have the capability of generating subtitles for an audio track playing on your phone, people with limited hearing now have a direct way to enjoy more content too.
Those people who have difficulties with vision can benefit greatly from a camera app with scene detection and text detection.
But even simple application such as a color picker on the camera can aid people who have color blindness in determining whether if a banana is ripe and more.
This section will highlight some of the benefits \acrshortpl{bci} can offer to people with disabilities.

\subsection{Parkinson's disease}
\label{subsec:introduction_helping_disabled_parkinson}
TODO

\subsection{Audiovisual aid}
\label{subsec:introduction_helping_disabled_audiovisual}
TODO

\subsection{Wheelchair users}
\label{subsec:introduction_helping_disabled_wheelchair}
TODO

\section{Small projects with big impact}
\label{sec:introduction_small_projects}
TODO

\subsection{TODO: example small project}
\label{subsec:introduction_small_projects_XXX}
TODO

\subsection{TODO: example small project}
\label{subsec:introduction_small_projects_YYY}
TODO

\subsection{Using a 3 signal system for basic controls}
\label{subsec:introduction_small_projects_ours}
TODO

\section{Ethical questions}
\label{sec:introduction_ethical}
TODO

\subsection{Risk of data mining for advertising}
\label{subsec:introduction_ethical_ads}
TODO

\subsection{Making the rich even richer}
\label{subsec:introduction_ethical_expensive}
TODO

\subsection{Confronting users with their brain}
\label{subsec:introduction_ethical_medical_users}
TODO