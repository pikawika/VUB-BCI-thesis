% TODO:
%   - XXX
% ----------  
% Questions:
%   - XXX

% zeggen dat voor bepalen van uw systeem dat ge wilt doen moet ge zien naar arnau zijn review: 5.2. A literature-based guide to DL for biosignal control

% Efforts are being made to develop standardized platforms that facilitate the deployment of control systems into real or simulated environments. Previous research solutions attempted to create common ecosystems for neurorobotic applications, in terms of open source frameworks such as the Neurorobotics platform (Falotico et al 2017), ROS-Health (Beraldo et al 2018) and its successor ROS-Neuro (Tonin et al 2019). ROS-Neuro was already succesfully used in a DL context (Valenti et al 2020, 2021). For more general applications, the OpenVibe platform provides several environments and integrates with a large variety of devices for BCI control experiments (Renard et al 2010).

% \Citet{bci_prostheses} also highlight that whilst multi-label classification of \gls{eeg} is possible with considerable accuracy in an offline lab setting, the number of detectable classes is limited in a real-time and real-life environment. uit bci_common_use_cases_prosthesis_exoskeleton

% feedback is belangrijk en kan soms voor actie zijn om dan te cancellen met bv eye movement oid

\chapter{Moving from an offline classification system towards an online BCI system}
\label{ch:online_bci_system}
TODO
